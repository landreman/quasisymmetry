\chapter{Overview}

This program constructs the shapes of stellarators that are quasisymmetric to first
order in the distance from the magnetic axis.
The method is based on the theory in \cite{GB1,GB2}
and the numerical method in \cite{PaperII}.
This latter paper is available in the Git repository for this program.



\section{Required libraries}

\begin{itemize}

\item {\ttfamily LAPACK} (for solving linear systems)
\item {\ttfamily MPI} (for parallelization of parameter scans)
\item {\ttfamily NetCDF} (for writing the output file)

\end{itemize}

Most of these libraries will be available on any high-performace computing system. {\ttfamily LAPACK}
is available on Apple Mac computers (as part of the Accelerate framework) if you install Xcode through the App store.

The {\ttfamily make test} scripts use \python~(version 2.7),
{\ttfamily numpy}, and {\ttfamily scipy}.
The plotting script {\ttfamily quasisymmetryPlotScan} uses the same \python~libraries
as well as {\ttfamily matplotlib}.

\section{Cloning the repository}

The source code for \quasisymmetry~is hosted in a {\ttfamily git} repository at
\url{https://github.com/landreman/quasisymmetry}.
You obtain the \quasisymmetry~source code by cloning the repository. This requires several steps.

\begin{enumerate}
\item Create an account on \url{github.com}, and sign in to {\ttfamily github}.
\item Click the icon on the top right to see the drop-down menu of account options, and select the ``Settings'' page.
\item Click on ``SSH and GPG keys'' on the left, and add an SSH key for the computer you wish to use. To do this, you may wish to read see the ``generating SSH keys'' guide which is linked to from that page: \url{https://help.github.com/articles/connecting-to-github-with-ssh/}
\item From a terminal command line in the computer you wish to use, enter\\
{\ttfamily git clone git@github.com:landreman/quasisymmetry.git}\\
 to download the repository.
\end{enumerate}

Any time after you have cloned the repository in this way, you can download future updates to the code by entering {\ttfamily git pull} from any subdirectory within your local copy.



\section{Single calculations vs. scans}

The code has two basic modes of operation, {\ttfamily general\_option = "single"} vs {\ttfamily "scan"}. 
When {\ttfamily general\_option = "single"}, a single set of input parameters (magnetic axis shape etc) is
considered, and a VMEC input file is generated. When {\ttfamily general\_option = "scan"},
the input parameters are scanned, and no VMEC input file is generated. For either option, results
are saved in a NetCDF file. The variables that are saved in the NetCDF file are different depending on
{\ttfamily general\_option}.

\section{Parallelization}

When computing the shape of a single configuration, the calculation is so fast (typically on the order of 1 ms)
that parallelization is unnecessary. However parallelization is useful when doing parameter scans,
in which case the calculation is `embarassingly parallel'. MPI is used for this latter case.


\section{\ttfamily make test}

To test that your \quasisymmetry~executable is working, you can run {\ttfamily make test}.  Doing so will run
\quasisymmetry~for some or all of the examples in the {\ttfamily examples/} directories.
After each example completes, several of the output quantities
will be checked, using the
{\ttfamily tests.py} script in the example's directory.
The {\ttfamily make test} feature is very useful when making changes to the code, since it allows you to check
that your code modifications have not broken anything and that previous results
can be recovered.

If you run {\ttfamily make retest},
no new runs of \quasisymmetry~will be performed, but the {\ttfamily tests.py} script
will be run on any existing output files in the \path{/examples/} directories.

\section{Units}

As in \vmec, all of \quasisymmetry's input and output parameters use SI units: meters, Teslas, Amperes, and combinations thereof.

\section{Plotting results}

The python program \quasisymmetryPlotScan~will display some of the results from a \quasisymmetry~parameter scan.

\section{Output quantities}

The output variables are documented using metadata (the `{\ttfamily long\_name}' attribute)
in the netCDF output files {\ttfamily quasisymmetry\_out.<extension>.nc}.
To view the available output variables, their annotations, and their values, you can run
{\ttfamily ncdump quasisymmetry\_out.<extension>.nc | less} from the command line.
Some of the most commonly used output quantities which can be found in the netCDF file are 
{\ttfamily iotas},
{\ttfamily max\_curvatures},
{\ttfamily max\_elongations},
{\ttfamily axis\_helicities},
{\ttfamily scan\_R0c},
{\ttfamily scan\_Z0s},
and
{\ttfamily scan\_eta\_bar}.

\section{Questions, Bugs, and Feedback}

We welcome any contributions to the code or documentation.
For write permission to the repository, or to report any bugs, provide feedback, or ask questions, contact Matt Landreman at
\href{mailto:matt.landreman@gmail.com}{\nolinkurl{matt.landreman@gmail.com} }






